\chapter*{Abstract}\label{abstract}
\ac{MDSD} has become more 
and more present since the last years because of the paradigm shift
from coding to modeling. 
One of the most popular and commonly used modeling languages is the \ac{UML}. It
provides the possibility of extending the modeling language itself by
making use of the implemented \textit{profiling} mechanism. New elements with its own semantics can be added to
existing elements of the \ac{UML} thus enabling integration of new domains or
features easily.
A popular example for exploiting this mechanism is the \ac{SysML}, which extends the \ac{UML}
in the domain of systems engineering applications.

In the area of text-based tools the parallel work
paradigm has been used for the last decades. In the area of
model-based tools there have only be solutions which do not facilitate the freedom
or the functionality on par with the latter, which leaves many
features known to be desired. Three of these are the matching of corresponding
elements, the detection and presentation of differences and the creation
followed by the application of patches between two models.
The \textit{SiDiff} tool facilitates the first, whereas the
\textit{SiLift} tool provides the remaining two features.

This Master's Thesis introduces the integration of \ac{UML} profiles in both
tools:\\
Supporting a broad range of modeling languages and domains is crucial to
modeling tools in practicality. Therefore the integration of \ac{UML} profiles
is a significant enhancement in this area and is done in both tools for
supporting the whole processing pipeline: Beginning at the creation
of differences and concluding at the application of a patch.
A real world industrial automation case study using \ac{SysML} as modeling
language is presented and used as an exemplary input model to demonstrate the
final result of this integration process.
