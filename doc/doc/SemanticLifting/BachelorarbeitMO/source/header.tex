%
% header.tex
%
\documentclass[%
	pdftex,%              PDFTex verwenden
	a4paper,%             A4 Papier
%	oneside,%             Einseitig
	twoside,%	      Zweiseitig
%	bibtotocnumbered,%    Literaturverzeichnis nummeriert einfügen
%	idxtotoc,%            Index ins Verzeichnis einfügen
%	halfparskip,%         Europäischer Satz mit abstand zwischen Absätzen
%	chapterprefix,%       Kapitel anschreiben als Kapitel
	headsepline,%         Linie nach Kopfzeile
%	footsepline,%         Linie vor Fusszeile
	11pt,%                Größere Schrift, besser lesbar am bildschrim
	openright,%			  Anfang immer auf rechter Seite 
	pointlessnumbers,%	  Nummerierung der Kapitel ohne abschließenden Punkt
%	smallheadings,%		  Kleinere Überschrifen
  headinclude,%
  ]{scrreprt}





%\usepackage{parskip}
%\usepackage{setspace}


\usepackage{url}


%
% Paket für die Indexerstellung.
%
\usepackage{makeidx}


%
% Paket für Übersetzungen ins Deutsche
%
%\usepackage[german, ngerman]{babel}
\usepackage[ngerman]{babel}
 
%
% Pakete um Latin1 Zeichnensätze verwenden zu können und die dazu
% passenden Schriften.
%
\usepackage[utf8]{inputenc}
\usepackage[T1]{fontenc}
%\usepackage{textcomp}
\usepackage{lmodern}  % Schriftart mit deutschen Umlauten

%
% Paket zum Erweitern der Tabelleneigenschaften
%
\usepackage{array}


\newcommand{\todo}[1]{\textbf{\textsc{\textcolor{red}{(TODO: #1)}}}}



%
% Paket um Grafiken einbetten zu können
%
\usepackage[pdftex]{graphicx}
\usepackage{wrapfig}

%
% Spezielle Schrift verwenden.
%
\renewcommand{\encodingdefault}{T1}
%\renewcommand{\familydefault}{pfr}
%\renewcommand{\sfdefault}{pfr}

%
% Spezielle Schrift im Koma-Script setzen.
%
% old:
%\setkomafont{sectioning}{\normalfont\bfseries}
%\setkomafont{llabel}{\normalfont\bfseries}
%\setkomafont{pagehead}{\normalfont\itshape}
%\setkomafont{descriptionlabel}{\normalfont\bfseries}
% new
%\setkomafont{sectioning}{\sffamily\bfseries}
%\setkomafont{captionlabel}{\sffamily\bfseries}
%\setkomafont{captionlabel}{\sffamily}
%\setkomafont{pagehead}{\sffamily\small\itshape}
%\setkomafont{pagehead}{\sffamily\small}
%\setkomafont{descriptionlabel}{\sffamily\bfseries}

%
% Zeilenumbruch bei Bildbeschreibungen.
%
\setcapindent{1em}


\usepackage[size=normalsize]{caption}


%Fussonotennummerierung durch das gesamte Dokument
\usepackage{chngcntr}
\counterwithout{footnote}{chapter}


%
% kopf und fusszeilen
%
%\usepackage[automark]{scrpage2}
%\pagestyle{scrheadings}
%\pagestyle{empty}

\usepackage{fancyhdr} 
\pagestyle{fancy}

\renewcommand{\chaptermark}[1]{\markboth{\MakeUppercase{\thechapter\ #1}}{}}

%
%% Kopf- und Fusszeilen formatieren
\fancypagestyle{normal}{
\fancyhf{}
\renewcommand\headrulewidth{0.4pt}
\fancyhead[RE]{\leftmark}
\fancyhead[LO]{\rightmark}
\fancyhead[RO,LE]{\thepage}
}
% Anfang eines neuen Kapitels etc.
\fancypagestyle{plain}{
\renewcommand\headrulewidth{0.0pt}
\fancyhf{}
\fancyfoot[CO]{\thepage}
}
% Header für das Inhaltsverzeichnis
\fancypagestyle{ihv}{
\fancyhf{}
\renewcommand\headrulewidth{0.4pt}
\fancyhead[RE,LO]{INHALTSVERZEICHNIS}
\fancyhead[RO,LE]{\thepage}
}
% Header für das Abkürzungsverzeichnis
\fancypagestyle{akv}{
\fancyhf{}
\renewcommand\headrulewidth{0.4pt}
\fancyhead[RE,LO]{ABKÜRZUNGSVERZEICHNIS}
\fancyhead[RO,LE]{\thepage}
}
% Header für den Anhang
\fancypagestyle{app}{
\fancyhf{}
\renewcommand\headrulewidth{0.4pt}
\fancyhead[RE,LO]{ANHANG}
\fancyhead[RO,LE]{\thepage}
}
% Header für das Abbildungsverzeichnis
\fancypagestyle{abv}{
\fancyhf{}
\renewcommand\headrulewidth{0.4pt}
\fancyhead[RE,LO]{ABBILDUNGSVERZEICHNIS}
\fancyhead[RO,LE]{\thepage}
}
% Header für das Literaturverzeichnis
\fancypagestyle{bib}{
\fancyhf{}
\renewcommand\headrulewidth{0.4pt}
\fancyhead[RE,LO]{LITERATURVERZEICHNIS}
\fancyhead[RO,LE]{\thepage}
}
% Header für den Index
\fancypagestyle{swv}{
\fancyhf{}
\renewcommand\headrulewidth{0.4pt}
\fancyhead[RE,LO]{INDEX}
\fancyhead[RO,LE]{\thepage}
}
% Header für Listings
\fancypagestyle{lst}{
\fancyhf{}
\renewcommand\headrulewidth{0.4pt}
\fancyhead[RE,LO]{LISTINGS}
\fancyhead[RO,LE]{\thepage}
}

%
% 2-seitiger Druck
%
%\rohead[\pagemark]{\pagemark}
%\rehead[\leftmark]{\leftmark}
%\lohead[\rightmark]{\rightmark}
%\lehead[\pagemark]{\pagemark}
%\ifoot[]{}
%\cfoot[]{}
%\ofoot[]{}
%
% 1-seitiger Druck
%
%\rohead[]{\pagemark}
%\rehead[]{}
%\lohead[]{\leftmark}
%\lehead[]{\pagemark}
%\ifoot[]{}
%\cfoot[]{}
%\ofoot[]{}
%\chead[]{}


%%Zeilenumbruch nach Paragraph
% \usepackage{titlesec}
% \titleformat{\paragraph}%
% {\bf}%
% {\theparagraph}%
% {0.5em}%
% {}


\usepackage{booktabs}
\usepackage{longtable}
\usepackage{multirow}

% \setlength\LTpre{-0.2cm}
% \setlength\LTpost{-1cm}


%%%%%%%%%%%%%%%%%%%%%%%%%%%%%
% Mathematik-Einstellungen

%
% mathematische symbole aus dem AMS Paket.
%
\usepackage{amsmath}
\usepackage{amssymb}
\usepackage{amsthm}

\usepackage{shadethm}

%\usepackage{thmbox}


% Equation Counter
\numberwithin{equation}{section}

% Definitionen
\newtheoremstyle{examplestyle}% name of the style to be used
  {10mm}% measure of space to leave above the theorem. E.g.: 3pt
  {10mm}% measure of space to leave below the theorem. E.g.: 3pt
  {}% name of font to use in the body of the theorem
  {}% measure of space to indent
  {\bfseries}% name of head font
  {\newline}% punctuation between head and body
  {10mm}% space after theorem head
  {}% Manually specify head

\theoremstyle{examplestyle}

\definecolor{shadethmcolor}{rgb}{.95,.95,.95}     % Farbe des Hintergrundes 
%\definecolor{shaderulecolor}{rgb}{0.0,0.0,1.0}   % Farbe des Rahmens
%\setlength{\shadeboxrule}{1pt}   		  % Breite des Rahmens

\newshadetheorem{definition}{Definition}[section]
\newshadetheorem{lemma}{Lemma}[section]
\newshadetheorem{beweis}{Beweis}[section]
\newshadetheorem{voraussetzung}{Voraussetzung}[section]

% Ende Mathematik-Einstellungen
%%%%%%%%%%%%%%%%%%%%%%%%%%%%%



%
% Paket um Textteile drehen zu können
%
\usepackage{rotating}
%\usepackage{lscape}

%
% Package für Farben im PDF
%
\usepackage{color}


%
% Seitenabmessungen
%
\setlength{\topmargin}{0pt}
\setlength{\voffset}{0cm}
\setlength{\textheight}{20.5 cm}
%\setlength{\headsep}{0.7cm}
\setlength{\evensidemargin}{1.15cm}
\setlength{\oddsidemargin}{0.2cm}
\setlength{\marginparwidth}{0.5cm}
\setlength{\textwidth}{14.7cm}
\setlength{\headwidth}{14.7cm}
\setlength{\skip\footins}{15mm}
%\setlength{\footskip}{2cm}


%
% Bis zu welcher Strukturtiefe soll nummeriert werden
%
\setcounter{secnumdepth}{2} 

%
% Hack um doublepages zu löschen
%
\makeatletter
\def\cleardoublepage{\clearpage\thispagestyle{empty}%
  \if@twoside \ifodd\c@page\else
  \hbox{}\newpage%
  \if@twocolumn\hbox{}\newpage\fi\fi\fi}
\makeatother

%
% Paket für Links innerhalb des PDF Dokumentes
%
% \definecolor{LinkColor}{rgb}{0,0,0.5}
\definecolor{LinkColor}{rgb}{0,0,0}
\usepackage[%
	pdftitle={A Rule-Based Approach to the  Semantic Lifting of Model Differences in the context of
	Model Versioning},%
	pdfauthor={Manuel Ohrndorf},%
	pdfcreator={},
	pdfsubject={Bachelorarbeit},
	pdfkeywords={Modelldifferenzen},
	plainpages=false,
	pdfpagelabels,
	bookmarksnumbered=true,
	bookmarksopen=true,
	pdfpagemode=UseOutlines]
	{hyperref}
\hypersetup{colorlinks=true,%
	linkcolor=LinkColor,%
	citecolor=LinkColor,%
	filecolor=LinkColor,%
	menucolor=LinkColor,%
	pagecolor=LinkColor,%
	urlcolor=LinkColor}

%
% Paket um Listings sauber zu formatieren.
%
\usepackage{listings}
\lstloadlanguages{TeX}

%
% Neue Sprachdialekte
%
%
% Pseudocode
%
\lstdefinelanguage{pseudo}{
	keywords={in, if, endif, else, for , endfor, function, return, break, true, false, when, each, new, boolean},
	sensitive=true,
	comment=[l]{//}
}

%              
% WORKAROUND, damit lstlistoflistings funktioniert:
% Quelle: http://www.komascript.de/node/477
%
\makeatletter% --> De-TeX-FAQ
\renewcommand*{\lstlistoflistings}{%
  \begingroup
    \if@twocolumn
      \@restonecoltrue\onecolumn
    \else
      \@restonecolfalse
    \fi
    \lol@heading
    \setlength{\parskip}{\z@}%
    \setlength{\parindent}{\z@}%
    \setlength{\parfillskip}{\z@ \@plus 1fil}%
    \@starttoc{lol}%
    \if@restonecol\twocolumn\fi
  \endgroup
}
\makeatother% --> \makeatletter

%
% Abkürzungsverzeichnis
%
\usepackage{nomencl}
% Befehl umbenennen in abk
\let\abk\nomenclature
% Deutsche Überschrift
\renewcommand{\nomname}{Abkürzungsverzeichnis}
% Punkte zw. Abkürzung und Erklärung
\setlength{\nomlabelwidth}{.20\hsize}
\renewcommand{\nomlabel}[1]{#1 \dotfill}
% Zeilenabstände verkleinern
\setlength{\nomitemsep}{-\parsep}
\makenomenclature

%
%Für Gedanken an der Seite => mit der letzten Zeile wird das wieder auskommentiert
%
\newcommand{\marginlabel}[1]{\mbox{}\marginpar{\raggedleft\hspace{0pt}#1}}
\newcommand{\Gedanke}[1]{\marginlabel{#1}}
%\renewcommand{\Gedanke}[1]{}           % outcomment this to get Gedanken!

%
% ---------------------------------------------------------------------------
% Listing Definationen für Code
%
% 



\definecolor{lbcolor}{rgb}{0.95,0.95,0.95}
\definecolor{commentcolor}{rgb}{0.4,0.4,0.4}
\lstset{language=pseudo,
	numbers=left,
	stepnumber=1,
	numbersep=5pt,
	numberstyle=\tiny,
%	breaklines=true,
% 	breakautoindent=true,
% 	postbreak=\space,
	tabsize=2,
	basicstyle=\small,
	commentstyle=\color{commentcolor}\itshape,
	stringstyle=\ttfamily,
	keywordstyle=\color{black}\bfseries,
	showspaces=false,
	showstringspaces=false,
	extendedchars=true,
	backgroundcolor=\color{lbcolor}}
%
% ---------------------------------------------------------------------------
%

%
% Zeilenabstand
%
\renewcommand{\baselinestretch}{1.2}

%% Schusterjungen und Hurenkinder vermeiden
%\clubpenalty = 10000 %Schusterjunge
%\widowpenalty = 10000 %Hurenkind


% Wird statt siehe bspw. die Abk. s. bevorzugt:
\renewcommand{\seename}{s.}

%
% Löst das Problem der franz. Anführungszeichen über verbatim-Umgebung
%
%\newcommand {\franzL} {\begin{verbatim}<<\end{verbatim}}
%\newcommand {\franzR} {\begin{verbatim}>>\end{verbatim}}
%\newcommand {\frq} {\begin{verbatim}<\end{verbatim}}
%\newcommand {\flq} {\begin{verbatim}<\end{verbatim}}

%
% Neue Umgebungen
% ---------------------------------------------------------------------------

\newenvironment{ListChanges}%
	{\begin{list}{$\diamondsuit$}{}}%
	{\end{list}}

%
% Index erzeugen
%
\makeindex

%
% EOF
%



%\usepackage{tabularx} 
%\usepackage{amsmath} 
% 
%\usepackage{multicol} 
%\usepackage{float} 
%\usepackage{amssymb} 
%
%\usepackage{pifont}  
%\include{codestyle} 