\textbf{Kurzfassung:}\hspace{0.3cm} Im Zentrum des \textit{Model-Driven Software Development} (MDSD)
stehen Modelle, welche häufig durch Teams entwickelt werden. Um in einem Team mit Modellen zu
arbeiten, werden Versionsverwaltungswerkzeuge benötigt. Eine Hauptaufgabe dieser Werkzeuge ist das
Vergleichen von verschiedenen Versionen oder Varianten eines Modells. Um zwei Modelle miteinander
zu vergleichen wird eine Differenz zwischen den beiden Modellen durch ein spezielles
Differenzwerkzeug berechnet. Da es eine Vielzahl von verschiedenen Modelltypen gibt, müssen die
Differenzwerkzeuge generisch arbeiten um mit allen Modelltypen umgehen zu können. Dadurch entsteht
allerdings das Problem, dass die Differenz zwischen zwei Modell Versionen nur auf Basis der internen
Repräsentation der Modelle angegeben werden kann. Die Differenzen auf Basis der internen Darstellung
werden hier als  low-level Änderungen bezeichnet. Solche low-level Änderungen sind aber für
Entwickler die nur die externe Repräsentation der Modelle kennen schwer zu lesen. Diese Arbeit setzt
sich damit auseinander die Lesbarkeit der Differenzen zu verbessern, indem den low-level Änderungen
wieder bestimmte Editieroperationen zugeordnet werden.
% Die Änderungen einer Editieroperation werden durch eine s.g. Editierregel vorgegeben. Das Auslesen
% der Editieroperation aus der Differenz wird durch die s.g. Erkennungsregeln erledigt. Eine
% Erkennungsregel sucht in der Differenz nach einem Muster von low-level Änderungen die zu der
% entsprechenden Editierregel passen. Anschließend werden die low-level Änderungen zu einer
% Editieroperation in einem s.g. Semantic-Change-Set gruppiert.
Dieser Prozess wird hier als Semantic-Lifting bezeichnet.

\vspace{1cm}

\textbf{Abstract:} \hspace{0.3cm} Models are in the center of \textit{Model-Driven Software
Development} (MDSD) and they are often developed in teams. To comply with the task of sharing a
model among team members we need special version management tools. One main task of a version
management tool is the comparison of different model versions. We also need special model difference
tools to calculate the changes between the two models. The algorithms of the difference tools have
to deal with many different model types. Instead of developing an individual algorithm for each
model type, the difference tools have to be generic. One problem of a generic difference algorithm
is that it has to compare the models based on their internal representation. Thus, difference tools
initially derive low-level changes from the internal representation of a model. But the low-level
changes are often incomprehensible for normal tool users, who usually prefer model changes to be
explained in terms of edit operations that are available from a user's point of view. So the primary
goal here is to increase the readability of the difference by lifting the low-level changes to the
level of user edit operations. We will call this process Semantic-Lifting.
