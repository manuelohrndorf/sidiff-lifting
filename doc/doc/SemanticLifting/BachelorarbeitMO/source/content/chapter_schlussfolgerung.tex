\chapter{Schlussfolgerung}
\label{schlussfolgerung}

In dieser Arbeit wurde ein Proof-Of-Concept des in \cite{KeKT2011ASE} beschriebenen Konzepts
durchgeführt. Dabei wurde für ein ausgewähltes Metamodell ein vollständiger Durchlauf durch alle
Phasen des Konzepts beschrieben. Von der Erstellung der Editierregeln und der automatischen
Generierung der entsprechenden Erkennungsregeln, über die Berechnung einer technischen Differenz,
bis hin zur vollständig semantisch gelifteten Differenz. 

Es wurde gezeigt, dass es möglich ist mit Hilfe von Henshin als Mustererkennungswerkzeug den
low-level Änderungen einer technischen Differenz wieder eine bestimmte Editieroperation zuzuordnen,
was in den Praxistests auch zu durchweg guten und vor allem aus Benutzersicht lesbaren Ergebnissen
geführt hat. Durch die Strukturierung der Differenz lässt sich auf den ersten Blick nachvollziehen
welche und wie viele Änderungsschritte an einem Modell durchgeführt wurden, ohne jede low-level
Änderung einzeln zu betrachten. Außerdem werden dem Benutzer einige grafische Werkzeuge an die Hand
gegeben, um die Semantic-Lifiting-Engine zu konfigurieren und um die Berechnung zu steuern.

In der Praxis hat sich auch gezeigt, dass der Algorithmus auf relativ gute Berechnungszeiten
kommt. Auf technischer Ebene wurde dies durch ein paar auf den Algorithmus angepasster
Optimierungen erreicht (Abschnitt \ref{filter} und \ref{optimierung}). Hier könnte aber bei
einer genaueren Analyse wahrscheinlich noch Potenzial für weitere Optimierungen gefunden werden. Zum
Beispiel durch eine genauere (ggf. statische) Analyse der Erkennungsregeln um nicht anwendbare
Regeln noch gezielter auszufiltern. Ggf. könnte auch eine Minimierung des Arbeitsgraphen auf dem die
Erkennungsregeln arbeiten einen Performance Gewinn bringen. Ein Großteil des Arbeitsgraphen wird
häufig für die Erkennung der Editieroperationen nicht benötigt.

Zwei Teile, die konzeptuell Probleme bereitet haben, waren zum einen das Liften von sequentiell
abhängigen Editieroperationen (Abschnitt \ref{sequential}) und zum anderen das Referenzieren von
importierten Modellen (Abschnitt \ref{resource_sets}). Letzteres könnte man auch als
technisches Problem ansehen. Für beide Probleme wurden aber auch Lösungsansätze geliefert. In der
Praxis tun diese in den meisten Fällen ihren Dienst. Aus theoretischer Sicht ist hier aber noch
etwas Analysebedarf. Ansonsten ließen sich die anderen Teile des Algorithmus wie im Konzept
\cite{KeKT2011ASE} beschrieben ohne letztendliche Probleme umsetzen.

Aufbauende Projekte auf dieser Arbeit, könnten vor allem sich an die Differenz Pipline 
(Abbildung \ref{fig:diff_pipeline}) anschließende Funktionalitäten sein. Neben der Gruppierung von
low-level Änderungen wäre es für den Benutzer sicherlich hilfreich, wenn die entsprechenden
Änderungen auch anschaulich visualisiert werden könnten. Um die gewonnen Differenzen dann
weiter zu verwerten, könnten sich aber auch typische Repository Funktionen, wie z.B. das Mergen von
Modellen auf Basis der einzelnen Editierschritte, anschließen. Der Vorteil bei Mergen auf
semantischer Ebene ist, dass ein dazu berechneter Patch auch auf Modelle angewendet werden kann die
sich vom Ausgangsmodell unterscheiden.

Ein weiterer Punkt der bei der Erstellung der Editierregeln (Abschnitt \ref{editierregeln}) auffällt
ist, dass die Erzeugung der Atomic-Editierregeln relativ schematisch von der Hand geht. In der
Praxis verbraucht diese Aufgabe aber selbst für kleinere Metamodelle einiges an Zeit und erfordert
außerdem ein genaues Verständnis des Metamodells. Um den Konfigurationsaufwand des Algorithmus zu
minimieren wäre es denkbar, dem Benutzer beim Erstellen der Editierregel einen Großteil der Arbeit
abzunehmen, indem man die Regeln soweit wie möglich automatisch aus dem Metamodell ableitet.

Insgesamt lässt sich sagen, dass die technischen Ergebnisse der Implementierung mit den erwarteten
aus dem Konzept \cite{KeKT2011ASE} übereinstimmen. Die Editieroperationen werden wieder erkannt,
wodurch sich die Komplexität der Differenz aus Benutzersicht stark verringert. Außerdem gibt es in
diesem Bereich bisher keine vergleichbaren Ansätze.
\begin{quote}
"`We are not aware of a generic algorithm which semantically lifts the low-level differences and
which can be adapted to a large variety of model types."' \cite{KeKT2011ASE} (S.10)
\end{quote}
Alles in allem lassen sich die aus diesem Projekt resultierenden Ergebnisse also durchaus als
vielversprechend bezeichnen.
