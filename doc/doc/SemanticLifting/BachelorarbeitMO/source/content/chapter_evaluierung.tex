\chapter{Evaluierung}
\label{evaluierung}

Für die Evaluierung wurde ein Ecore Modell aus dem Eclipse Modeling Framework (EMF) ausgewählt
(org.eclipse.emf.codegen.ecore). Verglichen wurden einmal die Versionen 1.0 bis 10.0 und in
5er Schritten die Versionen 1.0 bis 45.0. Die Ergebnisse der Semantic-Lifting Berechnungen sind in
folgender Tabelle \ref{eval} aufgelistet.

\begin{table}[h!]
\centering
\caption{GenModel Auswertung}
\begin{tabular}{lccccccc}
\hline
\textsc{Version} & SCS & GLL/ULL & KF & K & LS & ML & LL\\
\hline
\hline
01x02 & 7 & 10/0 & 1,4 & 105 & 2 & 345 & 797\\
02x03 & 6 & 24/0 & 4 & 106 & 1 & 147 & 411\\
03x04 & 1 & 4/0 & 4 & 112 & 1 & 122 & 223\\
04x05 & 1 & 4/0 & 4 & 113 & 1 & 129 & 104\\
05x06 & 1 & 1/0 & 2,6 & 114 & 1 & 113 & 81\\
06x07 & 9 & 23/0 & 4 & 114 & 2 & 120 & 464\\
07x08 & 1 & 4/0 & 4 & 121 & 1 & 126 & 111\\
08x09 & 1 & 4/0 & 4 & 122 & 1 & 146 & 240\\
09x10 & 1 & 4/0 & 4 & 123 & 1 & 144 & 110\\
\hline
01x05 & 15 & 42/0 & 2,8 & 105 & 2 & 393 & 1246\\
05x10 & 13 & 36/0 & 2,8 & 114 & 2 & 387 & 960\\
10x15 & 20 & 64/0 & 3,2 & 122 & 3 & 279 & 2779\\
15x20 & 13 & 46/0 & 3,5 & 136 & 3 & 248 & 1275\\
20x25 & 14 & 49/0 & 3,5 & 148 & 2 & 419 & 1823\\
25x30 & 6 & 21/0 & 3,5 & 161 & 2 & 213 & 675\\
30x35 & 17 & 64/0 & 3,8 & 167 & 2 & 214 & 3007\\
35x40 & 7 & 27/0 & 3,9 & 184 & 1 & 195 & 582\\
40x45 & 14 & 37/0 & 2,6 & 191 & 2 & 228 & 797\\
\hline
\end{tabular}
\label{eval}
\end{table}

 \begin{itemize}
   \item \textbf{SCS:} Anzahl der Semantic-Chang-Sets.
   \item \textbf{GLL/ULL:} Gruppierte / Ungruppierte low-level Änderungen.
   \item \textbf{KF:} Der mittlere Kompressionsfaktor  $\frac{SCS}{GLL}$.
   \item \textbf{K:} Anzahl der Korrespondenzen
   \item \textbf{LS:} Lifting Schritte zum 
   liften sequenziell abhängiger Editieroperationen.
   \item \textbf{ML:} Laufzeit von Matching und Differenz-Ableitung in \textit{ms}.
   \item \textbf{LL:} Lifting Laufzeit in \textit{ms}.
 \end{itemize}
 Was zunächst positiv auffällt ist, dass alle Editieroperationen erkannt wurden, so dass keine
 ungruppierten low-level Änderungen in der Differenz zurück bleiben. Die dadurch erreichte
 Kompression liegt im Mittel bei 3,2, was bei Ecore auch in etwa zu erwarten ist, da meistens 3
 low-level Änderungen anfallen, wenn ein Objekt mit Referenzen vom Container zum Objekt und vom
 Objekt zum Container eingefügt wird. 
 
 Was bei den Korrespondenzen auffällt ist, dass die Modelle in der Regel wachsen und nur selten
 Objekte entfernt werden. Auch sind die Anzahl der Änderungen von einer Modell Version zur
 nächsten in den meisten Fällen relativ klein. Um auch sequentiell abhängige Editieroperationen zu
 erkennen, werden für dieses Modell nie mehr als 3 Lifting Schritte benötigt, was zeigt, dass die
 Abhängigkeiten zwischen den Editieroperationen meistens nicht allzu komplex werden.
 
Die Laufzeit des Semantic-Lifting hängt hauptsächlich von der Anzahl der low-level Änderungen
 ab und nicht von der Anzahl der Korrespondenzen. Also auch nicht von der Größe des Modells. Dieses
 ist genau das Verhalten, welches die Optimierungen in Abschnitt \ref{optimierung} bewirken soll.
 Insgesamt sind die Ergebnisse also schon sehr nah an dem, was erwartet wurde.
