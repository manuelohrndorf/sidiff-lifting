\chapter{Recognition-Engine}
\label{recognition}

Die Recognition-Engine ist das eigentliche Kernstück des Semantic-Liftings. Durch das Aufrufen der
Recognition-Engine werden die Erkennungsregeln auf die technische Differenz angewendet, wodurch die
Semantic-Change-Sets erzeugt werden.

\section{Aufbauen des Henshin Arbeitsgraphen}

Bevor die Erkennungsregeln auf Differenz angewandt werden können muss zunächst der Henshin
Arbeitsgraph aufgebaut werden. Im Arbeitsgraphen müssen alle Objekte enthalten sein, auf die die
Erkennungsregeln abgebildet werden.

\begin{itemize}
  \item Die technische Modell Differenz, welche geliftet werden soll.
  \item Die Modell A und B Instanzen.
  \item Das Metamodell von Modell A und B (in unserem Fall Ecore) für das Matching der Typknoten.
  \item Ecore als Meta-Metamodell für das Matching des Typs der Typknoten (\texttt{EReference,
  EAttribute}).
\end{itemize}

\section{Filtern der Erkennungsregeln}
\label{filter}

In der Praxis hat sich gezeigt, dass die Differenzen zwischen zwei Modell Revisionen in der Regel
relativ klein sind. Aufgrund dieser Annahme lassen sich an dieser Stelle bereits einige
Erkennungsregeln ausfiltern, für die es keinen Match im Arbeitsgraphen geben kann. Zu diesem Zweck
werden zunächst die low-level Änderungen für jeden Typ in der Differenz gezählt. Im nächsten Schritt
werden dann die low-level Änderungsknoten jeder Erkennungsregel gezählt und mit der Anzahl der
jeweiligen low-level Änderungen in der Differenz verglichen. Ist die Anzahl der Änderungen in der
Erkennungsregel größer als in der Differenz, werden diese Regeln ausgefiltert. Enthält eine
Differenz z.B. keine Attribute-Value-Changes, dann werden alle Erkennungsregeln ausgefiltert, die
ein oder mehr Attribute-Value-Change Knoten enthalten.

\section{Optimierung der Erkennungsregeln}
\label{optimierung}

Die technische Abbildung einer LHS auf den Arbeitsgraphen wird in Henshin durch die Reihenfolge der
Knoten in der Liste des LHS-Graphen beeinflusst. Henshin sucht in der Reihenfolge, in der die Knoten
in der Liste liegen, nach passenden Objekten, auf die der Knoten abgebildet werden kann. Wurden die
ersten Knoten in der Liste bereits auf Objekte abgebildet, schränkt dies die Auswahl der möglichen
Objekte für weitere folgende Knoten ein. Es empfiehlt sich also, Knoten in der Liste nach vorne zu
ziehen, für die es einzeln betrachtet nur wenige Abbildungsmöglichkeiten gibt.

Wenn man davon ausgeht, dass ein Modell in der Praxis in den meisten Fällen zunehmend anwächst und
die Differenzen zwischen den einzelnen Revisionen relativ klein bleiben (also wenig Änderungen),
dann lassen sich für die Typknoten und Änderungsknoten am wenigsten Abbildungsmöglichkeiten finden.
Für die Correspondence-Knoten und Modell Knoten gibt es hingegen die meisten Möglichkeiten, diese
auf den Arbeitsgraphen abzubilden. Aufgrund dieser Überlegungen ergibt sich die folgende Reihenfolge der
Knoten im LHS-Graphen:

\begin{enumerate}
  \item \textbf{Differenzknoten:} Die Differenz als Wurzelknoten ist innerhalb des Arbeitsgraphen
  eindeutig, sie bringt allerdings keine wirkliche Einschränkung der Abbildungsmöglichkeiten für
  den Rest des Transformationsgraphen.
 
  \item \textbf{Typknoten:} Da der Typ einer Referenz oder eines Attributs über den Namen auf das
  Metamodell abgebildet wird, sind die Typknoten einzeln betrachtet nicht zwangsläufig eindeutig.
  Man kann aber davon ausgehen, dass innerhalb eines Modells die Anzahl der Referenzen und Attribute
  mit dem gleichen Namen in der Praxis eher gering ist.
 
  \item \textbf{Änderungsknoten:} Die Änderungsknoten werden zusätzlich anhand der Anzahl der
  Änderungen in der Differenz sortiert, wobei Änderungen mit geringerer Anzahl nach vorne gezogen
  und häufig vorkommende Änderungen in der Liste nach hinten sortiert werden.
 
  \item \textbf{Modell A und B Knoten:} An dieser Stelle werden die einzelnen low-level Änderungen
  über die Modell A bzw. B Objekte miteinander verbunden bzw. einander korrekt zugeordnet. Dies ist
  besonders dann entscheidend, wenn es mehrere low-level Änderungen gibt, deren Referenzen auf
  Objekte des gleichen Typs zeigen. Dies ist z.B. der Fall, wenn eine Editieroperation mehrfach auf
  verschiedene Objekte angewendet wurde.
 
  \item \textbf{Correspondence-Knoten:} Durch die Correspondences wird zum Schluss der Bezug
  zwischen Modell A und Modell B hergestellt.
  % Betrachten wir noch einmal unser Beispiel aus Abbildung \ref{fig:add_eclass}. 
  % -> Vllt besser Add EReference
  Beispielsweise beim Einfügen einer \texttt{EReference}  in ein Modell dient die \texttt{EClass}
  als Container Objekt. Für dieses Container Objekt muss nun eine passende Correspondence gefunden
  werden. Wurden für die Abbildung der Erkennungsregel bereits die Typ-, Änderungs- und Modell B
  Knoten gefunden, dann lässt sich die Correspondence (falls vorhanden) in diesem Fall sofort
  eindeutig festlegen.
  
  Umgekehrt wäre es sehr aufwändig (bei einem großen Modell) nacheinander alle Correspondences mit
  \texttt{EClasses} zu betrachten und dann für jede Correspondence zu überprüfen ob diese  durch 
  eine Add-Reference mit einem Add-Object verbunden ist. Daher sollten die Correspondeces erst zum
  Schluss des Matchings auf den Arbeitsgraphen abgebildet werde.
\end{enumerate}
Auf diese Weise hängt die Berechnungszeit von der Anzahl der Änderungen im Modell ab. Ansonsten
würde die Berechnungszeit mit zunehmender Größe der Modelle immer länger werden, was zu einer
sehr schlechten Skalierbarkeit des Algorithmus im Bezug auf wachsende Modell Revisionen führen
würde.

Da man nicht mit Sicherheit davon ausgehen kann, dass die Reihenfolge der Knoten zwischen
Generieren, Serialisieren und Laden gleich geblieben ist, werden die Knoten erst direkt vor dem
Matching sortiert.

\section{Ausführen der Erkennungsregeln}

Das Ausführen der Erkennungsregeln lässt sich in drei Phasen unterteilen:

\begin{enumerate}
  \item \textbf{Parallelisierungsphase:} Die erste Phase ist eine rein technische Optimierung. Sie
  basiert darauf, dass die Erkennungsregeln unabhängig voneinander auf die Differenz abgebildet und
  angewendet werden können. Um dies auszunutzen, soll die Matching- und Anwendungsphase
  parallelisiert werden. Dazu werden die Regeln immer in Blöcken an einzelne Threads übergeben.
  Die Block-Größe, also die Anzahl der Regeln pro Thread, kann in der Recognition-Engine
  konfiguriert werden. Bei Mehr-Kern-Systemen lies sich dadurch eine deutlich bessere Auslastung der
  CPU erreichen, da sich die Last durch die Aufteilung effektiver auf die einzelnen CPU-Kerne
  verteilen lässt.
  
  \item \textbf{Matchingphase:} In dieser Phase wird die linke Seite jeder Regel (LHS) auf den
  Arbeitsgraphen abgebildet. Wodurch die s.g. \textit{Matches} erzeugt werden. Ein Match
  gibt genau den Teil des Arbeitsgraphen an, der in der Anwendungsphase durch die rechte Seite der
  Regel (RHS) ersetzt wird. Ein Match für eine Regel muss nicht eindeutig sein. D.h. für
  jede Regel können mehrere und zum Teil auch sich überschneidende Matches existieren. Im
  Idealfall entspricht die Anzahl der Matches für jede Erkennungsregel der Häufigkeit, mit
  der die Editierregel auf das Modell angewendet wurde. Welche Probleme sich durch Überschneidungen
  von Matches ergeben und wie diese behandelt werden, wird im nächsten Abschnitt 
  \ref{post_processing} Post-Processing beschprochen.
  
  \item \textbf{Anwendungsphase:} Wie bereits erwähnt werden in der Anwendungsphase alle Matches
  durch die entsprechende rechte Seite der Erkennungsregel ersetzt. Der einzige Teil, der nicht im
  Schnitt von LHS und RHS enthalten ist, ist das Semantic-Change-Set und die Kanten zu den
  jeweiligen low-level Änderungen. Das ist eben genau der Teil, welcher der Differenz hinzugefügt
  wird. D.h. nach diesem Schritt sind die Semantic-Change-Sets mit den dazu gehörigen low-level
  Änderungen in der Differenz enthalten. Da sich allerdings ggf. die Matches überschneiden können,
  überschneiden sich jetzt auch die entsprechenden Semantic-Change-Sets. D.h.
  eine low-level Änderung kommt dann in mehreren Semantic-Change-Sets vor.
\end{enumerate}
