\section{Matching}
\label{sec:matching}

Als Matching wird in diesem Zusammenhang der Prozess bezeichnet, in dem die beiden Modelle A und B
miteinander verglichen werden. Dazu werden zustandsbasierte Vergleichsalgorithmen verwendet, welche
eine symmetrische Differenz auf der Basis des aktuellen Zustands der beiden Modelle berechnen.  Ziel
des Matchings ist es herauszufinden welche Elemente in beiden Modellen übereinstimmen. Diese
Übereinstimmungen werden als Korrespondenzen (\textit{engl. Correspondence}) zwischen Modellen A und
B bezeichnet. \cite{KeKT2011ASE} (S.3)

\begin{quote}
\centering
$A \Delta B := \{x \mid (((x \in  A)\wedge(x \not\in B))\vee((x \in B)\wedge(x \not\in A))) \}$
\end{quote}

\begin{figure}[htb]
  \centering
  \includegraphics[scale=0.4]{images/symmetrische_Differenz.png}
  \caption{Symmetrische Differenz}
  \label{fig:sym_diff}
\end{figure}

Die Korrespondenzen geben zunächst die Schnittmenge zwischen den Modellen an. Für die Berechnung der
korrespondierenden Elemente gibt es grundsätzlich verschiedene (zustandsbasierte) Techniken und
Algorithmen:

\begin{itemize}
  \item \textbf{ID-basiert:} In diesem Ansatz wird davon ausgegangen, dass jedes Modell-Element eine
  eindeutige statische ID besitzt. D.h. diese ID wird beim Erzeugen des Elements vergeben und darf
  sich auch im weiteren Entwicklungsverlauf des Modells nicht verändern. Der Hauptvorteil dieses
  Ansatzes besteht darin, dass keine spezielle Konfiguration für unterschiedliche Modelltypen nötig
  ist. Außerdem lassen sich Korrespondenzen sehr effizient berechnen. Dieser Ansatz funktioniert
  allerdings nicht für Modelle, die unabhängig voneinander erzeugt wurden. Außerdem muss die Vergabe
  von IDs durch die verschiedenen Entwicklungsumgebungen gewährleistet sein. \cite{KPRP2009} (S.2)
  
  \item \textbf{Signaturbasiert:} In diesem Ansatz wird eine Signatur zum vergleichen der
  Modell-Elemente verwendet. Im Gegensatz zu einer ID ist die Signatur nicht statisch. Eine Signatur
  wird dynamisch aus den Werten der Eigenschaften eines Elements berechnet. Daher können auf diese
  Weise auch Modelle verglichen werden, die unabhängig voneinander entwickelt wurden. Dieser
  Ansatz ist allerdings mit deutlich mehr Aufwand verbunden, da für jeden Modell-Element-Typ eine
  Berechnungsfunktion für die jeweilige Signatur angelegt werden muss. \cite{KPRP2009} (S.3)
  
  \item \textbf{Ähnlichkeitsbasiert:} In diesem Ansatz werden Modelle als typisierte, attributierte
  Graphen behandelt. Die einzelnen Modell-Elemente werden dabei durch die Summe ihrer Eigenschaften
  identifiziert. Dabei sind nicht alle Eigenschaften gleich wichtig. Hierzu wird in den meisten
  Algorithmen eine Konfiguration benötigt die das relative Gewicht der einzelnen Eigenschaften
  angibt. Herauszufinden welche Gewichtung der Eigenschaften für eine Modellierungssprache ein
  möglichst optimales Ergebnis liefert, ist allerdings häufig ein "`trial-and-error"' Prozess.
  Außerdem können generische Ansätze keine Rücksicht auf die Semantik der Modellierungssprache
  nehmen, welche die Genauigkeit verbessern und die Anzahl der individuellen Vergleiche verringern
  würde. \cite{KPRP2009} (S.3)
  
  \item \textbf{Sprachspezifisch:} Ein sprachspezifischer Matching Algorithmus wurde speziell für
  eine Modellierungssprache implementiert. Da diese Algorithmen die Semantik ihrer
  Modellierungsprache kennen ist hier in der Regel auch kein Semantic-Lifting notwendig. Daher wird
  hier auch nicht näher auf solche Algorithmen eingegangen. Der Nachteil dieses Ansatzes bestehen
  im hohen Entwicklungsaufwand und der auf die Modellierungssprache eingeschränkten Benutzbarkeit.
\end{itemize}
Zur Berechnung der Korrespondenzen für das Semantic-Lifting kann grundsätzlich ein beliebiger
Algorithmus verwendet werden. Die Korrespondenzen müssen nur immer in die interne Darstellung
konvertiert werden. (Siehe \texttt{Correspondeces} Abbildung \ref{fig:diffmodel}.) Ein Vergleich
verschiedener Ansätze ist in \cite{KPRP2009} zu finden. Im folgenden wird nun auf die im Rahmen
dieser Arbeit verwendeten Algorithmen eingegangen.

% Beim Vergleichen von Modellen spielt immer auch die Struktur eine wichtige Rolle. In Ecore werden
% Modelle in Form einer Baumartigen Aggregationsstruktur aufgebaut. Die einzelnen Elemente dieser
% Struktur sind typisiert und ggf. attributiert. Dabei können Elemente dieser Struktur hinzugefügt
% oder entfernt werden. Im Sinne einer Struktur kann man auch von Verschiebung einzelner Elemente
% sprechen, wenn ein Element aus einem Teilbaum einem anderen zugeordnet wird. Die Verschiebung äußert
% sich in einem Modell durch das verändern der Referenzen zwischen Container und Element. Ist das
% Element kein Blatt in der Baumstruktur so können auf diese Weise auch ganze Teilbäume verschoben
% werden. In vielen Fällen kann es wünschenswert sein ein Verschieben als solches zu erkennen und
% nicht als löschen und einfügen von Elementen. In diesem Fall würden dann Korrespondenzen zwischen
% den Verschobenen Elementen bzw. Teilbäumen angelegt, nicht aber zwischen den veränderten Referenzen
% der Container und  der Elemente bzw. Wurzeln der Teilbäume. Korrespondenzen können außerdem
% zwischen nicht gleichen Elementen auftreten. Zum Beispiel kann ein Element auf Grund seiner
% strukturellen Anordnung als Korrespondierend angesehen werden, möglicherweise wurden aber Attribute
% wie der Name des Elements verändert. Auf Grund dieser verschiedenen Eigenschaften muss
% die Menge der Korrespondenzen je nach verwendetem Matching Algorithmus nicht immer Eindeutig sein.

\subsection{SiDiff}
\label{sidiff}

SiDiff ist ein an der Universität-Siegen entwickelter Metamodell unabhängiger Ansatz, um Modelle zu
vergleichen. Der Algorithmus ist hauptsächlich darauf ausgelegt Modell-Elemente nach ihrer
Ähnlichkeit zu vergleichen (ähnlichkeitsbasiert), unterstützt aber auch Ansätze zum ID-basierten
oder signaturbasierten Modellvergleich. Der Hauptvorteil von SiDiff besteht darin, dass dem
Benutzer eine sehr frei konfigurierbare Umgebung geboten wird, die auf jeden Modelltyp angepasst
werden kann. Der Modelltyp muss sich lediglich als graphähnliche Struktur darstellen lassen. Der
generische Algorithmus muss für jeden Modelltyp konfiguriert werden. Eine Konfiguration besteht aus
einer Transformation vom Originaldokument in die interne Repräsentation, einer Definition für die
Gewichtung der Eigenschaften der zu vergleichenden Modell-Elemente für die Berechnung der
Ähnlichkeiten und einer Spezifizierung der auszugebenden Daten. \cite{SiDiff}

% SiDiff is an meta model-independent approach to model comparison. It is primarily based on the
% notion of similarity between model elements, but covers other approaches like id-based or
% signature-based model comparison as well. The main advantage of SiDiff is that it offers a highly
% configurable environment and is therefore easily adaptable to any model type where models can be
% represented in a graph-like structure
% 
% The generic algorithm must be configured for each type of model. A configuration consists of a
% transformations from the original document into an internal structure, the definition of the
% similarity of elements, a specification of the output, and further details.

\subsection{Named-Element-Matcher}
\label{nem}

Der Named-Element-Matcher wurde im Rahmen diese Projekts als Testwerkzeug verwendet, er vergleicht
die beiden Modelle A und B einfach anhand der Namen der einzelnen Elemente. Der Matcher eignet sich
vor allem für selbst konstruierte Testfälle, da das Ergebnis des matchings relativ gut abzusehen
ist. Dieser Algorithmus lässt sich als ein signaturbasierter Algorithmus einordnen. Als Signatur
wird in diesem Fall eben nur der Namen benutzt.
