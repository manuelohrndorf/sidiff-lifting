\chapter{Erstellen der Editierregeln}
\label{editierregeln}

Um Modelle bearbeiten zu können, benötigen wir darauf ausgelegte Werkzeuge. In der Regel werden dazu
grafische Editoren oder Transformationstechniken benutzt. Welche Technik zur Bearbeitung eines
Modells verwendet wurde und welche Semantik hinter den einzelnen Veränderungen im Modell steckt, ist
zum Zeitpunkt der Differenzberechnung nicht mehr direkt nachzuvollziehen. Das Problem ist, dass dem
generischen Differenzberechner keine Informationen über die Semantik der möglichen
Editieroperationen vorliegt. Um die Semantik wiederherzustellen, müssen wir allerdings wissen,
welche Editieroperationen für einen bestimmten Modelltyp existieren und wie diese aussehen. Hierzu
werden s.g. Editierregeln verwendet. Die Editierregeln sind die Grundlage für das spätere Semantic-Lifting.
Eine Editierregel beschreibt eine Editieroperation, die vom Entwickler auf einem Modell eines
bestimmten Typs ausgeführt werden kann. Dabei hat jeder Modelltyp auf Basis seines Metamodells und der
Semantik andere Editieroperationen. Daher müssen auch die Editierregeln für jeden Modelltyp einzeln
angegeben werden.

\section{Identifikation von Editierregeln}

Zunächst stellt sich an dieser Stelle die Frage, welche Editierregeln benötigt werden, um das
vollständige Bearbeiten eines Modells zu ermöglichen. Grundsätzlich können hier natürlich
verschiedene Ansätze gewählt werden. Eine Möglichkeit wäre es, falls vorhanden, den Editor für den
bestimmten Modelltyp zu analysieren und festzustellen, welche Editieroperationen zur Verfügung
gestellt werden. Eine weitere Möglichkeit besteht darin, das eigentliche Metamodell und die damit
verbundenen Constraints zur Erstellung der Editierregeln zu verwenden.

Da das Semantic-Lifting ein generischer Ansatz ist, kann zu diesem Zeitpunkt noch keine Aussage über
die später in der Praxis verwendeten Metamodelle getroffen werden. Die eigentliche Entscheidung,
welche Editierregeln benötigt werden und wie diese aussehen, bleibt damit zum Teil auch dem
Entwickler überlassen, der die Semantic-Lifting-Engine für ein bestimmtes Metamodell
konfiguriert. Dabei sollte bedacht werden, dass eben nicht immer eindeutig klar ist, auf welche Art
das Modell bearbeitet wurde. Unterschiedliche Editoren können für ein und dasselbe Metamodell
unterschiedliche Editieroperationen haben.

Die Editierregeln müssen also alle Editieroperationen nachbilden, die grundlegend möglich wären.
Trivial aber entscheidend an dieser Stelle ist, dass später nur solche Editieroperationen geliftet
werden, für die auch eine Editierregel oder eine entsprechende Zerlegung von mehreren Editierregeln
existiert. Aufgrund dieser Überlegungen erschien eine Analyse des Metamodells zur Ableitung der
Editierregeln am sinnvollsten. An Hand des Metamodells lässt sich am besten schematisch und formal
nachvollziehen, welche Editierregeln benötigt werden. Allgemein formuliert werden zunächst die
Editierregeln benötigt, mit denen für ein Modell neue Elemente erzeugt, bestehende Elemente entfernt
oder ggf. abgeändert werden können. Ziel ist es eine Basismenge an Editierregeln zu erzeugen, die
sich an folgende Definition anlehnt. Statt \textit{"`atomic pattern"'} wird hier im Folgenden der
Begriff der \textbf{Atomic-Editierregel} verwendet:

\begin{quote}
"`Atomic patterns constitute basic elements for composing modeling patterns. They are governed by
two characteristics: On the one hand, they maintain model consistency, of course. On the other hand,
they cannot be split into smaller modeling patterns keeping model consistency. Therefore, they form
the smallest units of consistent model changes.

By using atomic patterns only, it is possible to create all eligible instances of a specific
modeling language. E.g., for each model element defined by the modeling language, at least one
particular atomic modeling pattern exists performing the minimum of required modeling steps to
create this model element. Otherwise this model element is no eligible element of the modeling
language, e.g. due to prohibition by constraints."' \cite{W2010} (S.11)
\end{quote}

\section{Editierregeln für Ecore}

Betrachtet man das Ecore Metamodell (Abbildung \ref{fig:ecore_metamodel}), so lassen sich 69
Atomic-Editier"-regel identifizieren. Abstrakte Klassen (nicht deren Attribute und Referenzen),
abgeleitete Referenzen und nicht veränderbare Attribute müssen nicht betrachtet werden. Die Regeln
sollen wie in der Definition beschrieben, die Konsistenz des Modells aufrecht erhalten. Man muss
bedenken, dass die Ecore Editoren auch Editieroperationen zulassen, die das Modell inkonsistent
werden lassen. Zum Beispiel können Klassen ohne Namen oder Attribute ohne Typ angelegt werden. Die
Atomic-Editierregeln lassen sich wie unten aufgelistet in mehrere Kategorien einteilen. Die
Aufteilung in diese Kategorien dürfte sich aber in der Regel auch auf jeden anderen Modelltyp
übertragen lassen.

\begin{itemize}
  \item \textbf{11 $\times$ Add <Class>:} Hinzufügen eines Modell-Elements.
  \item \textbf{11 $\times$ Remove <Class>:} Entfernen eines Modell-Elements.
  \item \textbf{11 $\times$ Move <Class>:} Verschieben eines bestehenden Modell-Elements.
  \item \textbf{5 $\times$ Add Reference <Class> <Reference>:} Hinzufügen einer optionalen Referenz zu
  einem bestehenden Modell-Element.
  \item \textbf{5 $\times$ Remove Reference <Class> <Reference>:} Entfernen einer optionalen Referenz zu
  einem bestehenden Modell-Element.
  \item \textbf{3 $\times$ Change Reference <Class>:} Verändern einer benötigten Referenz eines
  bestehenden Modell-Elements.
  \item \textbf{23 $\times$ Attribute Value Change <Class> <Attribute>:} Ändern eines Attributs eines
  bestehenden Modell-Elements.
\end{itemize}
Regeln, die nicht in die Kategorie der Atomic-Editierregel fallen, werden im Folgenden als
\textbf{Advanced-Editierregel} bezeichnet. Eine Advanced-Editierregel kann im Prinzip als eine
Zusammensetzung bzw. Spezialisierung von Atomic-Editierregeln betrachtet werden.

% Der Editieranteil einer Advanced-Editierregel muss sich in mehrere Atomic-Editierregel zerlegen
% lassen. D.h. eine Advanced-Editierregel kann ggf. noch zusätzliche Einschränkungen dahingehend
% enthalten, dass ein bestimmter Zustand des Modells gegeben sein muss damit die Regel angewendet
% werden kann.

% 18 Advanced-Edit-Rules:
% 
% \begin{itemize}
% \item Create/Remove referenced EClass
% \item Create/Remove sub EClass
% \item Create/Remove super EClass
% \item Move EAttribute to neigbour
% \item Pull up EAttribute
% \item Push down EAttribute
% \item Rename EAttribute/EClass/EDataType/EEnum/EEnumLiteral/EOperation/
% EPackage/EParameter/EReference
% \end{itemize}

\section{EMF-Refactor}

Um die Editierregeln zu formulieren werden Henshin Transformationssysteme (TS) verwendet. Diese
Transformationssysteme müssen bestimmten Konventionen entsprechen, damit sie später für das
Semantic-Lifting weiterverarbeitet werden können. Die Konventionen entsprechen denen von
EMF-Refactor Regeln. EMF-Refactor ist ein Werkzeug um benutzerdefinierte Refactorings für Ecore
basierte Modelle anzulegen. Die einmal angelegten Refactorings können dann über den baumbasierten
EMF-Editor auf das Modell angewendet werden.  Der Unterschied zwischen einem Refactoring und einer
Editieroperation besteht darin, dass ein Refactoring nur die Struktur des Modells verändert, nicht
aber das Verhalten wie bei einer Editieroperation. Der Aufbau von Editierregeln für
Editieroperationen unterscheidet sich aber nicht von dem für Refactorings, daher lässt sich das
Konzept problemlos übernehmen. EMF-Refactor definiert drei Henshin Transformationssysteme, aus denen
ein Refactoring besteht.

\begin{itemize}
  \item \textbf{Initial-Check-TS} (\textit{Precondition})\textbf{:} Hier werden grundlegende Tests
  durchgeführt, ob das Refactoring angewendet werden darf. Dazu wird für jede Situation, in der das
  Refactoring nicht angewendet werden darf, eine Regel formuliert. In diesen Regeln wird genau die
  Situation dargestellt, die nicht auftreten darf. Das Refactoring darf nur dann ausgeführt werden,
  wenn keine Initial-Check-Regel zutrifft, bzw. anwendbar ist. Dazu wird eine Independent-Unit
  verwendet, die alle Regeln enthält. Eine Independent-Unit ist dann anwendbar, wenn genau eine
  Regel der Unit anwendbar ist. D.h. die Independent-Unit darf zum ausführen des Refactorings nicht
  anwendbar sein. Theoretisch können auch mehrere Units verschachtelt werden. Um dem EMF-Refactor
  kenntlich zumachen, welche Unit beim Initial-Check ausgeführt werden soll, muss diese Unit mit dem
  Namen \textit{mainUnit} gekennzeichnet werden.
  
  \item \textbf{Final-Check-TS}  (\textit{Precondition})\textbf{:} Der Final-Check unterscheidet
  sich strukturell nicht vom Initial-Check. Auch hier muss es eine Unit mit Namen \textit{mainUnit} 
  geben, die nicht anwendbar sein darf, damit das Refactoring durchgeführt werden kann. Im
 Unterschied zum Initial-Check wird der Final-Check durchgeführt, nachdem die für das Refactoring
 benötigten Daten vom Benutzer abgefragt wurden. Hier wird überprüft, ob die eingegebenen Daten für
 das Refactoring verwendet werden können. Zum Beispiel könnte beim Erstellen einer Klasse überprüft
 werden, ob eine Klasse mit dem gleichen Namen schon existiert.
 
  \item \textbf{Execute-TS:} In der Execute Phase wird das eigentliche Refactoring ausgeführt. Die
  in diesem Transformationssystem enthaltenen Regeln geben an, welche Änderungen an dem Modell
  durchgeführt werden sollen. Auch hier muss wieder eine \textit{mainUnit} spezifiziert werden.
\end{itemize}
Für das Semantic-Lifting ist hauptsächlich das Execute-TS von Bedeutung. Das Refactoring für eine
bestimmte Modellierungssprache wird hier auf Basis des Metamodells angegeben. Daher lassen sich aus
den Regeln die eigentlichen low-level Änderungen ablesen, die durch das Semantic-Lifting wieder
erkannt werden sollen. Im Fall des Initial- und Final-Check gehen wir zunächst einmal davon aus,
dass, wenn ein angewandtes Refactoring erkannt wurde, auch die Vorbedingungen
(\textit{engl. precondition}) für dieses Refactoring erfüllt waren.

Damit die Henshin Transformationssysteme durch die Semantic-Lifting-Engine verarbeitet werden
können, müssen diese den EMF-Refactor Konventionen entsprechen. Genauere Angaben und Beispiele sind
auf \cite{ERe} zu finden.
