%*************************************************************************
\section{SiLift benutzen}
%*************************************************************************
%************************
\subsection{Vergleich von Modellen}
%************************
Die Bedienung von SiLift als Vergleichswerkzeug soll am folgenden Beispiel demonstriert werden.
Ausgangsbasis sind die \textit{Ecore-Modelle} in Abbildung \ref{classdiagram_example}.

\begin{figure}[h!]
\centering
\includegraphics[width=\textwidth]{lifting/graphics/ecore-classdiagram_example.png}
\caption{Ecore-Modelle}
\label{classdiagram_example}
\end{figure}

Dabei stellt \texttt{model\_vb} das Ausgangsmodell und \texttt{model\_v1} das im Laufe eines Entwicklungsprozesses veränderte Modell dar.
Die entscheidenden Änderungen sind zum einen das Attribut \texttt{name}, welches durch den Entwicklungsprozess in die nun \textit{abstrakte} Klasse \texttt{Person} verschoben wurde und das neue Attribut \texttt{section} in der Klasse \texttt{Developer}.\\
Als nächstes selektieren Sie die beiden \textit{ecore}-Files im \textit{Package Explorer} und öffnen mit der rechten Maustaste das Kontextmenü.
Wählen Sie \texttt{SiLift} $\triangleright$ \texttt{Compare with each other} aus (vgl. Abb. \ref{silift-contextmenu_compare}).

\begin{figure}[h!]
\centering
\includegraphics[width=0.7\textwidth]{lifting/graphics/silift-contextmenu_compare.png}
\caption{SiLift über das Kontextmenü starten}
\label{silift-contextmenu_compare}
\end{figure}

Es öffnet sich ein Wizard-Dialog, der sich über zwei Seiten erstreckt und mehrere Konfigurationsmöglichkeiten bietet (Abb. \ref{silift-wizard_compare_page1} und \ref{silift-wizard_compare_page2}).
Um diese besser zu verstehen, folgt ein kleiner Exkurs über die Architektur von \textit{SiLift}.

\begin{figure}[h!]
\centering
\includegraphics[width=0.5\textwidth]{lifting/graphics/silift-wizard_compare_page1.png}
\caption{Einstellungen für das Erstellen gelifteter Differenzen: Seite 1}
\label{silift-wizard_compare_page1}
\end{figure}

\begin{figure}[h!]
\centering
\includegraphics[width=0.5\textwidth]{lifting/graphics/silift-wizard_compare_page2.png}
\caption{Einstellungen für das Erstellen gelifteter Differenzen: Seite 2}
\label{silift-wizard_compare_page2}
\end{figure}

%*************************************************************************
\section{SiLift-Architektur}\label{silift_architecture}
%*************************************************************************
Mit SiLift können Differenzen von \textit{EMF-basierten} Modellen, d.h. Modelle die auf dem \textit{Ecore-Metamodell} basieren, semantisch geliftet werden. Basierend auf einer gelifteten Differenz lassen sich \textit{Patches} bilden, sowie Modelle mischen.\\
Für \textit{domainspezifische} Modellierungssprachen bedeutet das, dass deren Metamodell (vgl. \ref{subsec:metamodel}) zuerst in ein entsprechendes \textit{Ecore-Modell} übertragen, sowie ein entsprechender \textit{Matcher} (vgl. \ref{sec:own_matching_engine}) und \textit{Technical Difference Builder} (vgl. \ref{sec:TechnicalDifferenceBuilder}) bereit gestellt werden müssen, bevor Editierregeln implementiert und Erkennungsregeln abgeleitet werden können.\\
Dieser Abschnitt führt die \textit{SiLift-Pipline} ein und dient als Grundlage der folgenden Abschnitte.


\begin{figure}[H]
\centering
\includegraphics[width=\textwidth]{architecture/graphics/silift-processing_pipeline.png}
\caption{SiLift Processing Pipeline}
\label{silift-processing_pipeline}
\end{figure}

Die Vorgehensweise von SiLift lässt sich am besten mit einer vierstufigen \textit{Pipeline}, wie in Abbildung \ref{silift-processing_pipeline} dargestellt,  vergleichen.
Als Eingabe dienen immer zwei Versionen eines Modells:

\begin{enumerate}

\item \textbf{Matching}: 
Aufgabe eines \textit{Matcher} ist es, die korrespondierenden Elemente aus Modell A und Modell B, also die Elemente, die in beiden Modellen übereinstimmen, zu identifizieren.
Dabei ist das Ergebnis vor allem davon abhängig anhand welcher Kriterien der Matcher eine Übereinstimmung festlegt.
Hier wird unter anderem unterschieden zwischen \textit{ID-}, \textit{signatur-} und \textit{ähnlichkeitsbasierten} Verfahren.\\
In SiLift stehen standardmäßig folgende \textit{Matcher-Engines} zur Verfügung:

\begin{itemize}
	\item \texttt{EcoreID Matcher}: Ein \textit{ID-basierter} Matcher (nutzt Werte von Attributen, die im Metamodell als ID-Attribute deklariert sind).
	\item \texttt{EMF Compare}: 
	Unterstützt alle drei Verfahren. \texttt{EMF Compare} kann unter \texttt{Win\-dow} $\triangleright$ \texttt{Preferences}: \texttt{EMF Compare} konfiguriert werden. \footnote{Informationen zum \texttt{EMF Compare Project} finden Sie unter \url{http://www.eclipse.org/emf/compare}.}
	
	\item \texttt{NamedElement Matcher}: 
	Ein \textit{signaturbasierter} Matcher, welcher die ent\-sprech\-en\-den Korrespondenzen anhand der Werte der jeweiligen Namensattribute bestimmt.
	
	\item \texttt{URIFragment Matcher}: 
	Ein \textit{signaturbasierter} Matcher, welcher die ent\-sprech\-en\-den Korrespondenzen anhand der Werte der \textit{Uri} der Elemente bestimmt (z.B. \texttt{eType=}"'\texttt{ecore:EDataType http://www.eclipse.org/emf/2002""/Ecore""\#//EString}"').
	
	\item \texttt{UUID Matcher}: Ein \textit{ID-basierter Matcher} (basiert auf XMI-IDs der XMI-Repräsentationen der Modelle, falls vorhanden).
\end{itemize}

Diese Liste ist keineswegs abgeschlossen und kann durch zusätzliche Matching-Engines, wie z.B. \textit{SiDiff} oder auch eigener Matcher ergänzt werden (siehe Abschnitt \ref{sec:own_matching_engine}). \\

\item \textbf{Difference derivation}: 
Ausgehend von den gefunden Korrespondenzen berechnet der \textit{Difference Derivator} eine technische Differenz (\textit{low-level difference}) der Mo\-del\-le.
Alle Objekte und Referenzen, für die keine Korrespondenz existiert müssen demnach entweder in Modell B hinzugefügt, oder aus Modell A entfernt worden sein.

\item \textbf{Semantic Lifting}:\label{page:semantic_change_sets}
Die zuvor berechnete technische Differenz enthält alle Än\-der\-ung\-en  auf Basis des Metamodells.
Diese sollen nun semantisch geliftet werden.
Bei einer \textit{semantisch gelifteten Differenz} handelt es um eine halbgeordnete Menge von auf einem vorhandenen Modell (dem Basismodell) ausgeführten \textit{Editieroperationen}.
Durch das liften der technischen Differenz werden die einzelnen Änderungen mit Hilfe von \textit{Erkennungsregeln} (engl. \textit{recognition rules}) in sogenannte \textit{Semantic Change Sets} gruppiert. Diese repräsentieren wiederum jeweils eine vom Benutzer ausgeführte Editieroperation.
Das Verhalten einer Editieroperation wird durch die zugehörige \textit{Editierregel} definiert, aus denen sich mit Hilfe des \textit{Recognition Rules Generators} die Erkennungsregeln ableiten lassen. 
Was wiederum eine gültige bzw. sinnvolle Editierregel ist hängt zum einem vom Metamodell, zum anderen von den Benutzerpräferenzen ab. 
Daher lassen sich die Editierregeln und somit auch die Erkennungsregeln grob zweier sogenannter Regelbasen (engl. \textit{Rule Bases}) zuordnen:

\begin{itemize}
\item \textbf{\texttt{Atomic Rule Base}}: 
Atomare Regeln umfassen das Erzeugen (engl. \textit{create}), Löschen (engl. \textit{delete}), Verschieben von Elementen (engl. \textit{move}) sowie das Ändern von Attributwertenv(engl. \textit{change}).
Sie lassen sich nicht in kleinere Teile zerlegen, ohne dass deren Anwendung zu einem inkonsistenten Modell führen würde.\\
Atomaren Regeln können mit Hilfe eines \textit{Editrulegenerators}\footnote{Weitere Information zum Editrulegenerator finden Sie unter \url{http://pi.informatik.uni-siegen.de/Projekte/SERGe.php}.} direkt aus dem Metamodell abgeleitet werden. 
Problematisch wird es, wenn weitere Restriktionen (engl. \textit{Constraints}), wie sie bspw. die UML in Form von \textit{OCL-Ausdrücken} benutzt, die Konsistenzkriterien eines Modells bzw. dessen Elemente weiter eingrenzen. 
I.d.R. werden diese nicht bei der Implementierung eines Metamodells berücksichtigt.
Hier bleibt nur die Möglichkeit die Regeln manuell zu editieren bzw. anzupassen.

\item \textbf{\texttt{Complex Rule Base}}: 
Die komplexen Editierregeln setzen sich i.d.R. aus den atomaren und anderen komplexen Regeln zusammen und beschreiben umfangreichere Editieroperationen, welche vor allem beim \textit{Refactoring} auftreten. 
Da solche Refactorings sehr benutzerspezifisch sind müssen komplexe Regeln generell von Hand erstellt werden.
\end{itemize}

\item \textbf{Difference Presentation UI}:
SiLift stellt zwei Benutzerschnittstellen (engl. \textit{User Interfaces}) zur Verfügung, um die semantisch gelifteten Differenzen anzuzeigen: 
einen Baum-basierten  und einen grafischen Editor, in dem die Differenzen \textit{gehighlightet} werden.\footnote{Beispielansichten finden Sie im \textbf{SiLift - Benutzerhandbuch für Endanwender}}
\end{enumerate}


Über den Wizard kann man an jeder Position der Pipeline eingreifen und somit das Verhalten von SiLift beeinflussen. Zusätzlich lassen sich noch folgende Einstellungen vornehmen:
%Was jetzt noch fehlt, sind die Optionen \textit{Select source model}, \textit{Merge Imports}, \textit{Recognition-Engine} und der \textit{Comparison mode}:

\begin{itemize}

\item \textbf{Select source model}: Die Differenzberechnung zwischen zwei Modellen ist nicht kommutativ.
I.d.R. handelt es sich bei den zu vergleichenden Modellen um unterschiedliche Revisionen ein und desselben Modells.
In den meisten Fällen wird man die ältere Revision (Modell VB) mit der neueren (Modell V1) vergleichen wollen.
Dennoch kann auch der andere Fall eintreten.
In \textit{Select source model} können Sie die Richtung der Differenzberechnung festlegen.\\
Zusätzlich kann man die Modelle vor der Differenzbildung noch validieren (vgl. Abb. \ref{silift-wizard_compare_page1}).

%\item \textbf{Merge Imports}: Diese Option ist in der aktuellen Version entfernt worden.
%Modelle können andere Modelle importieren und auf Objekte dieser referenzieren. Ist dies der Fall muss diese Option aktiviert sein, damit auch diese Modelle bzw. die referenzierten Objekte berücksichtigt werden.

\item \textbf{Scope}: Ein Modell muss nicht in sich geschlossen sein, sondern kann auf andere Modelle bzw. deren Elemente verweisen.
Mit Hilfe des Scopes kann man festlegen, ob diese Modelle bei der Erzeugung der Differenz ignoriert (\texttt{Single resource}), oder mit in diese aufgenommen werden sollen (\texttt{Complete resourceset}, vgl. Abb. \ref{silift-wizard_compare_page1}).


\item \textbf{Recognition-Engine}: Mit der Wahl einer \textit{Recognition-Engine} wird festgelegt, ob und wie die technischen Differenzen geliftet werden.
Wie der Name bereits andeutet, wird bei der Auswahl von \texttt{No Semantic Lifting} keine semantische Differnz erzeugt.
Für das Erzeugen einer semantischen Differenz stehen zum einen die \texttt{Simple Recognition Engine}, zum anderen die \texttt{Post Processed Recognition Engine} zur Verfügung.
Der Unterschied liegt im Auftreten von Überlappungen der \textit{Semantic Change Sets}, die vor allem bei der zusätzlichen Verwendung von komplexen \textit{Rule Bases} auftreten.
Wenn Sie komplexe Erkennungsregeln nutzen (und auch sonst) ist es daher ratsam die \texttt{Post Processed Recognition Engine} zu nutzen, um eben diese Überlappungen zu vermeiden (vgl. Abb. \ref{silift-wizard_compare_page2}).

\end{itemize}

Nachdem Sie nun die Konfigurationsmöglichkeiten kennengelernt haben wird es Zeit \textit{SiLift} auf die zuvor erstellten Modelle anzuwenden.
In unserem Beispiel ist \texttt{company\_vb} unsser Basismodell. Des Weiteren wählen wir den \texttt{NamedElement Matcher} und deaktivieren zunächst die komplexen Erkennungsregeln (vgl. Abb. \ref{silift-tutorial_compare_config}).

\begin{figure}[h!]
\centering
\includegraphics[width=0.8\textwidth]{lifting/graphics/silift-tutorial_compare_config.png}
\caption{Einstellungen für das Erstellen einer gelifteten Differenz ohne komplexe Erkennungsregeln}
\label{silift-tutorial_compare_config}
\end{figure}

Das Ergebnis wird in der \texttt{company\_vb\_x\_company\_v1\_NamedElement\_lifted\_post\-pro\-cessed.symmetric} gespeichert und lässt sich mit dem \textit{Difference Model Editor} öffnen (vgl. Abb. \ref{silift-tutorial_compare_atomic_lifted}).

\begin{figure}[h!]
\centering
\includegraphics[width=0.8\textwidth]{lifting/graphics/silift-tutorial_compare_atomic_lifted.png}
\caption{\texttt{company\_vb\_x\_company\_v1\_NamedElement\_lifted\_post\-pro\-cessed.symmetric}}
\label{silift-tutorial_compare_atomic_lifted}
\end{figure}

Jedes \textit{Sematic Change Set} steht für eine \textit{Editieroperation}, welche im Laufe des Entwicklungszyklus auf Modell VB angewandt wurde.
Somit werden Ihnen die Differenzen nun auf eine intuitive Weise präsentiert, ohne dass Sie das Metamodell in alle Einzelheiten kennen müssen.
Durch das Aufklappen eines Change Sets können jedoch die jeweiligen technischen Differenzen weiterhin angezeigt werden (vgl. Abb. \ref{silift-tutorial_compare_atomic_lifted}).
Zusätzlich werden die gefundenen Korrespondenzen aufgelistet.\\
Neben dem baumbasierten Editor lassen sich die Differenzen mittels eines graphischen Editors anzeigen.
Dieser lässt sich, wie in Abbildung \ref{silift-tutorial_compare_arrange_compare_view} dargestellt, aufrufen.

\begin{figure}[h!]
\centering
\includegraphics[width=0.6\textwidth]{lifting/graphics/silift-tutorial_compare_arrange_compare_view.png}
\caption{Aufruf des graphischen Editors (Compare View)}
\label{silift-tutorial_compare_arrange_compare_view}
\end{figure}

Durch Auswahl eines Change Sets werden die betroffenen Elemente in den Diagrammen \textit{gehighlightet} (vgl. Abb. \ref{silift-tutorial_compare_compare_view}).

\begin{figure}[h!]
\centering
\includegraphics[width=\textwidth]{lifting/graphics/silift-tutorial_compare_compare_view.png}
\caption{SiLift Compare View: vb nach v1 (nur atomare Editieroperationen)}
\label{silift-tutorial_compare_compare_view}
\end{figure}

Wenn wir uns nochmals die oben beschriebenen Änderungen ins Gedächtnis rufen, so lassen sich diesen die Editieroperationen wie folgt zuordnen:

\begin{itemize}
\item \texttt{Move Attribute To Another Class}: Verschiebe Attribut \texttt{name} von \texttt{Developer} nach \texttt{Person}.
\item \texttt{Create Attribute}: Erstelle neues Attribut \texttt{section} in \texttt{Developer}.
\item \texttt{SET EClass Abstract}: Die Klasse \texttt{Person} ist nun \textit{abstrakt}.
\item \texttt{Delete Attribute}: Entferne Attribut \texttt{name} aus \texttt{Manager}.
\end{itemize}

Nochmal zur Erinnerung: Die geliftete Differenz aus Abbildung \ref{silift-tutorial_compare_compare_view} wurde mit Hilfe der atomaren Erkennungsregeln erstellt.
Diese werden wiederum aus den atomaren Editierregeln abgeleitet.
Eine atomare Editierregel kann nicht in noch kleinere Regeln aufgeteilt werden, ohne dass deren Anwendung zu einem inkonsistenten Modell führen würde.
Diese Regeln umfassen i.d.R. das Erstellen (\texttt{create}), Entfernen (\texttt{remove}) und Verschieben (\texttt{move}) von Modellelementen sowie das Ändern von Attributwerten (\texttt{set}).\\
Betrachten wir die beiden Change Sets \texttt{Delete Attribute} und \texttt{Move Attribute To Another Class}.
In diesem Szenario wurde das Attribut \texttt{name} in der Klasse \texttt{Manager} gelöscht und aus der Klasse \texttt{Developer} nach \texttt{Person} verschoben.
Gleichzeitig ließe sich diese Differenz der Modelle jedoch auch als ein \textit{Refactoring} der Vererbungsbeziehung verstehen, indem übereinstimmende Attribute der Unterklassen in die Oberklasse verschoben wurden.
Die für ein solches Refactoring erforderliche Erkennungsregel umfasst also mehrere atomare Regeln.
Um solche Refactorings zu erkennen starten wir SiLift nun zusätzlich mit der komplexen \textit{Rule Base} (vgl. Abb. \ref{silift-tutorial_compare_config_complex}). Die restlichen Einstellungen können Sie aus Abbildung \ref{silift-tutorial_compare_config} übernehmen.

\begin{figure}[h!]
\centering
\includegraphics[width=0.5\textwidth]{lifting/graphics/silift-tutorial_compare_config_complex.png}
\caption{Einstellungen für das Erstellen einer gelifteten Differenz mit komplexen Erkennungsregeln}
\label{silift-tutorial_compare_config_complex}
\end{figure}

Das Ergebnis ist eine geliftete Differenz die anstatt vier nur noch drei Change Sets beinhaltet  (Abb. \ref{silift-tutorial_compare_compare_view_complex}).
Hier wurden die atomaren Regeln \texttt{Delete Attribute} und \texttt{Move Attribute To Another Class} durch die komplexe Regel \texttt{Pull Up Attribute} ersetzt.

\begin{figure}[h!]
\centering
\includegraphics[width=\textwidth]{lifting/graphics/silift-tutorial_compare_compare_view_complex.png}
\caption{SiLift Compare View: vb nach v1 (incl. komplexen Editieroperationen)}
\label{silift-tutorial_compare_compare_view_complex}
\end{figure}

Wie bereits erwähnt setzen sich komplexe Regeln aus atomaren und anderen komplexen Regeln zusammen.
Betrachten wir Abbildung \ref{silift-tutorial_compare_atomic_vs_complex}, so deckt die komplexe Regel \texttt{Pull Up Attribute} alle technischen Differenzen der beiden atomaren Regeln \texttt{Delete Attribute} und \texttt{Move Attribute To Another Class} ab.

\begin{figure}[h!]
\centering
\includegraphics[width=\textwidth]{lifting/graphics/silift-tutorial_compare_atomic_vs_complex.png}
\caption{Vergleich atomarer und komplexer Erkennungsregeln}
\label{silift-tutorial_compare_atomic_vs_complex}
\end{figure}

Durch komplexe Erkennungsregeln lassen sich somit größere Refactorings auf eine intuitive Weise darstellen.\\
